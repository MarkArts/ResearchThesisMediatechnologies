%!TEX root = ../Thesis.tex
\chapter{Conclusie}
\label{ch:conclusie}
\lhead{\emph{Conclusie}}

Om het ontwikkelen van interactieve 3D omgevingen mogelijk te maken voor niet-programmeurs is de VRInteractions plugin ontwikkeld. De plugin maakt het mogelijk om met een aantal simpele stappen de omgeving \gls{vr} klaar te maken. Door de \gls{vr} logica in C++ te schrijven en te koppelen aan Blueprints kunnen niet-programmeurs na drie workshops zelfstandig interactie toevoegen. Om de niet-programmeurs te ondersteunen en complexe problemen op te lossen is de beschikbaarheid van een programmeur nodig.

Met de plugin en beschreven workflow zijn de volgende projecten succesvol afgerond:

\begin{itemize}
	\item Een fly-trough door een menselijk lichaam
	\item Een appartement waarvan o.a. het meubilair en de vloer dynamisch aangepast kon worden
	\item Een demo van een machine uit een fabriek
	\item Een virtuele omgeving van een tentoonstelling
	\item Een leeromgeving rond het colosseum
	\item Een demo omgeving van de VRInteractions plugin
\end{itemize}