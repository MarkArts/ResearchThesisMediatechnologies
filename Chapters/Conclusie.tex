%!TEX root = ../Thesis.tex
\chapter{Conclusie}
\label{ch:conclusie}
\lhead{\emph{Conclusie}}

Om het ontwikkelen van interactieve 3D omgevingen mogelijk te maken voor niet-programmeurs is de VRInteractions plugin ontwikkeld. De plugin maakt het mogelijk om met een aantal simpele stappen \gls{vr} omgevingen op te zetten en om bestaande omgevingen werkend te krijgen in \gls{vr}.

Door het ontwikkelen met een workflow die prioriteit geeft aan vrijheid, experimenten en gebruiksgemak zijn de componenten iteratief ontwikkeld in samenwerking met de niet-programmeurs. Hierdoor is een intuïtieve koppeling gemaakt tussen Blueprints en \gls{vs}.

Op basis van de gemaakte projecten en de taken van een programmeur hierin zijn de volgende componenten ontwikkeld:

\begin{itemize}
	\item Een game mode voor een demo waarin de speler kan rondlopen en een voor een fly-through demo
	\item Een speler klasse per game mode
	\item Een component die verantwoordelijk is voor de input van de speler
	\item Een component die verantwoordelijk is voor de camera van de speler
	\item Een component die events afvuurt als er naar een object gekeken word
	\item Een HUD die het mogelijk maakt om op basis van de look events informatie te tonen
	\item Een 3D menu die zijn opties in een cirkel toont
	\item Een object wat verplaats kan worden door middel van ernaar te kijken.
\end{itemize}

Door de interactie te beperken tot kijken naar objecten is er voor gezorgd dat alle componenten in de VRInteractions plugin werken op de Oculus, Vive en de GearVR.

Met behulp van de plugin zijn meerdere projecten ontwikkeld door niet-programmeurs wat aantoont dat er succesvol een ontwikkelomgeving gerealiseerd is waarin niet-programmeurs in \gls{ue4} unieke virtual reality ervaring kunnen creëren.