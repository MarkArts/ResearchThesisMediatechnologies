%!TEX root = ../Thesis.tex
\chapter{Resultaten en Discussie}
\label{ch:resultatenEnDiscussie}
\lhead{\emph{Resultaten en Discussie}}

In dit hoofdstuk word er antwoord gegeven op de deelvragen en worden de resultaten in context geplaatst en besproken.

\section{Resultaten}
In Hoofdstuk~\ref{ch:BlueprintsEnCpp} is er geprobeerd antwoord te geven op de vraag "Hoe kan er een koppeling met \gls{vs} gemaakt worden die intuïtief is voor niet programmeurs?". Met de workflow die in paragraaf~\ref{sec:workflow} uitgezet is zijn de projecten uit paragraaf~\ref{sec:welkeComponenten} ontwikkeld en is er antwoord gegeven op de vraag "Welke componenten zullen gemaakt moeten worden?", 
namelijk de componenten uit de VRInteractions plugin die besproken is in hoofdstuk~\ref{ch:vrinteractions}. De componenten werken op basis van kijken naar objecten, een interactie vorm die werkt op de platforms die gekozen zijn in paragraaf~\ref{subsec:platforms}, waarmee antwoord gegeven is op de vraag "Is het mogelijk de componenten cross-platform te maken?".

Het gebruiksgemak van de VRInteractions plugin is getest, bijlage~\ref{appendix:usertest1}, na het geven van drie workshops, bijlages~\ref{appendix:workshop1}, \ref{appendix:workshop2} en \ref{appendix:workshop3}. Uit de test bleek dat het niet mogelijk was voor de niet-programmeurs om zelfstandig een complex interactief object te maken met behulp van de VRInteractions plugin.

\section{Discussie}
Al werd tijdens de gebruikstest van de VRInteractions plugin een onacceptabele resultaat behaald zijn de opdrachten uit workshop 3 wel succesvol zelfstandig gemaakt. Ook zijn er parallel aan het schrijven van deze scriptie twee projecten met de VRInteractions plugin ontwikkeld door de niet-programmeurs waarbij geen hulp nodig was, namelijk een demo in het Colosseum en het menu van een fly-through door het menselijk lichaam. Beide tester gaven aan dat ze een tijd, twee en drie weken, niet met Blueprints gewerkt hadden en dat ze daarom moeite hadden met de test. 

Naast het gebruik van de VRInteractions plugin waren er ook een aantal performance problemen in de demo's die niet opgelost konden worden door de niet-programmeurs. Al is het mogelijk voor niet-programmeurs om zelfstandig interactie toe te voegen aan een 3D omgeving zijn er toch problemen die alleen door een programmeur opgelost kunnen worden.

Uit de behoefte aan hulp voor de niet-programmeurs en het niet kunnen vermijden van problemen die de niet-programmeurs niet zelf kunnen oplossen word er geadviseerd om in een ontwikkel team minstens een programmeur met 3D ervaring beschikbaar te hebben die complexere problemen kan oplossen en ondersteuning kan geven aan de niet-programmeurs.