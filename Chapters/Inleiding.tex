%!TEX root = ../Thesis.tex
\chapter{Inleiding}
\lhead{\emph{Inleiding}}

In een periode van 18 weken zijn er een reeks van basis componenten voor de \gls{ue4} ontwikkeld met als doel niet-programmeurs efficiënt interactie aan \gls{vr} demo’s te laten toevoegen.

De \gls{ue4} is gebruikt omdat deze een uitgebreid \gls{vs} systeem heeft die het makkelijk maakt om simpele logica eenvoudig uit te drukken (zie hoofdstuk~\ref{ch:visualscripting}). Het \gls{vs} systeem maakt het mogelijk voor niet-programmeurs om simpele logica toe te voegen zonder dat een programmeur nodig is. Dit zorgt voor een hogere productiviteit en kwaliteit \cite{Cutumisu200732}.

\todo{quote source}

\section{Probleemstelling}

Momenteel is het toevoegen van interactieve elementen in \gls{vr} een intensieve programmeer taak omdat standaard nog geen interactie aanwezig is, en elke hardware weer andere input gebruikt. Het is daardoor lastig om de geprogrammeerde logica te hergebruiken in een ander project. Hierdoor is er altijd een programmeur nodig om de gewenste functionaliteit toe te voegen of aan te passen. 

In gameontwikkeling zijn er de afgelopen jaren steeds meer technieken ontwikkeld om van de afhankelijkheid van programmeurs te verminderen \cite{Cutumisu200732, ambientbehav}. De oplossing van \gls{ue4} is het gebruik van Blueprints om interactie in een spel te programmeren. Maar op het moment van schrijven is er nog geen koppeling tussen Blueprints en \gls{vr} interactie.

\todo{zo hebben x scriptiease gemaakt en b bla geprobeerd}

\section{Doelstelling}

De doelstelling van deze scriptie is het creëren van een reeks basis componenten voor \gls{ue4} waarmee niet programmeurs interactieve \gls{vr} demo’s kunnen maken die toegevoegde waarde hebben aan de workflow van DPI.

Aan het eind van dit traject zal het mogelijk zijn om zonder de hulp van een programmeur met de gemaakte componenten een bestaande omgeving geschikt te maken voor \gls{vr} en om een nieuwe \gls{vr} omgeving op te zetten.

\section{Onderzoeksvraag}

\subsection{Hoofdvraag}

De hoofdvraag van deze scriptie is: Hoe realiseren we een ontwikkelomgeving waarin niet-programmeurs, zoals 3D modellers en level designers, een efficiënte unieke \gls{vr} ervaring kunnen creëren in \gls{ue4}.

\subsection{Deelvragen}

\begin{itemize}  
\item Hoe kan er een koppeling met Virtual Scripting (\gls{vs}) gemaakt worden die intuïtief is voor niet programmeurs? 
	\begin{itemize}
	\item Wat is de huidige kennis van computer logica onder de werknemers van DPI?
	\item Is er extra training nodig om met de ontwikkel omgeving aan de slag te gaan?
	\end{itemize}
\item Welke componenten zullen gemaakt moeten worden?
	\begin{itemize}
	\item Wat zijn de huidige taken van een programmeur in het maken van een virtuele omgeving?
	\item Welke tools bestaan er al?
	\item Wat zijn de juiste abstracties van de componenten? 
	\end{itemize}
\item Is het mogelijk de componenten cross-platform te maken?
	\begin{itemize}
	\item Welke platforms zijn relevant?
	\item Wat zijn de verschillende mogelijkheden van deze platforms?
	\end{itemize}
\end{itemize}

\subsection{Onderzoeksmethoden}
\label{subsec:onderzoeksmethoden}

\subsubsection{Hoe kan er een koppeling met Virtual Scripting gemaakt worden die intuïtief is voor niet programmeurs?}

De vrijheid en mogelijkheden die de niet-programmeur krijgen, oftewel de abstractie van de componenten, zijn afhankelijk van de bestaande intuïtie van programmeren. Als een gebruiker bijvoorbeeld snapt hoe de “als dit dan dat” constructie werkt kan er meer vrijheid in de componenten gecreëerd worden. Maar als de gebruiker deze intuïtie niet heeft zal meer vrijheid, abstractie, alleen voor onduidelijkheid zorgen.

In de eerste plaats zal er onderzocht moeten worden wat de huidige kennis is van de niet-programmeurs. Dit kan door middel van interviews.

Vervolgens zal er onderzocht moeten worden wat een haalbare moeilijkheidsgraad is en of er baat is bij een extra training voor niet-programmeurs.

Als er sprake is van toegevoegde waarde van een extra training dan zal deze parallel aan het project gegeven worden en iteratief ontwikkeld worden.

\subsubsection{Welke componenten zullen gemaakt moeten worden?}
Het onderzoek naar welke componenten gemaakt moeten worden zal beginnen met een onderzoek naar de huidige taken van de programmeur tijdens het maken van een virtuele reality omgeving. Aan de hand hiervan zal er gekeken worden welke taken geautomatiseerd kunnen worden of versimpeld naar \gls{vs}.

Als er een duidelijk beeld is van de benodigde componenten zal er een onderzoek gedaan worden naar bestaande tools die ingezet kunnen worden om de workflow te verbeteren.

Als het duidelijk is welke componenten zelf geschreven moeten worden en hoeverre de niet-programmeurs met \gls{vs} om kunnen gaan, kan er bepaald worden welke abstractie de componenten zo breed inzetbaar mogelijk houdt. Met als randvoowaarde dat het intuïtief blijft voor de niet-programmeurs.

\subsubsection{Is het mogelijk de componenten cross-platform te maken?}
Om te bepalen of het mogelijk is de componenten cross-platform te maken zal er onderzocht worden welke platforms relevant zijn en wat de technische mogelijkheden voor elk platform zijn.

Daarna zal voor elke component een beslissing genomen moeten worden of er een alternatieve versie moet komen of dat het cross-platform in de component zelf meegenomen kan worden.
