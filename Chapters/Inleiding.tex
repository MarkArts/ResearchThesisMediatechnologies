\chapter{Inleiding}

In een periode van 18 weken zal ik proberen een reeks van basis componenten voor de Unreal Engine 4 te ontwikkelen waarmee niet-programmeurs efficiënt interactie aan virtual reality demo’s kunnen toevoegen.

De Unreal Engine zal gebruikt worden omdat deze een uitgebreid VS systeem heeft die het makkelijk maakt om simpele logica, zoals wanneer en waar iets moet plaats vinden, makkelijk uit te drukke. Het VS systeem maakt het mogelijk voor niet-programmeurs om simpele logica zelf toe te voegen zonder dat hier een programmeur voor nodig is. Naast een efficiëntere workflow maakt dit het ook makkelijk om als designer, 3D artist te experimenteren.

\section{Probleemstelling}

Momenteel is het toevoegen van interactieve elementen in virtual reality een intensieve programmeer taak omdat het vaak ontwikkelt wordt voor een specifieke use case. Het is daardoor vaak lastig om de geprogrammeerde logica te hergebruiken in een ander project. 

Hierdoor is er altijd een programmeur nodig om de gewenste functionaliteit toe te voegen of aan te passen. Vooral in gameontwikkeling zijn er veel sprongen gemaakt in het oplossen van dit probleem. Een van deze oplossingen is het gebruikt van “VS” om interactie in een spel te programmeren. Er zijn hier al veel sprongen in gemaakt maar het implementeren van VR gerelateerde componenten is nog niet standaard aanwezig.

\section{Doelstelling}

Het creëren van een reeks basis componenten voor Unreal Engine 4 waarmee niet programmeurs interactieve virtual reality demo’s kunnen maken die toegevoegde waarde hebben aan de workflow van DPI.

Aan het eind van dit traject zal het mogelijk zijn om met de gemaakte componenten een bestaande omgeving zonder programmeren geschikt te maken voor VR en om een nieuwe omgeving op te zetten zonder behulp van een programmeur.

\section{Onderzoeksvraag}

\subsection{Hoofdvraag}

Hoe realiseren we een ontwikkel omgeving waarin niet-programmeurs, zoals 3D artists en level designers, efficiënt unieke virtual reality ervaring kunnen creëren in Unreal Engine 4.

\subsection{Deelvragen}

\begin{itemize}  
\item Hoe kan er een koppeling met VS gemaakt worden die intuïtief is voor niet programmeurs. 
	\begin{itemize}
	\item Wat is de huidige kennis van computer logica onder de werknemers van DPI
	\item Is er extra training nodig om met de ontwikkel omgeving aan de slag te gaan
	\end{itemize}
\item Welke componenten zullen gemaakt moeten worden?
	\begin{itemize}
	\item Wat zijn de huidige taken van een programmeer in het maken van een Virtuele omgeving
	\item Welke tools bestaan er all
	\item Wat zijn de juiste abstracties van de componenten 
	\end{itemize}
\item Is het mogelijk de componenten cross-platform te maken 
	\begin{itemize}
	\item Welke platformen zijn relevant?
	\item Wat zijn de verschillende mogelijkheden van deze platformen
	\end{itemize}
\end{itemize}

\subsection{Onderzoeksmethoden}

\subsubsection{Hoe kan er een koppeling met VS gemaakt worden die intuïtief is voor niet programmeurs?}

De hoeveelheid abstractie van de componenten zijn afhankelijk van de bestaande intuïtie voor programmeren. Als een gebruiker namelijk snapt hoe de “als dit dan dat” constructie werkt kan er meer vrijheid in de componenten gecreëerd worden.

Als eerst zal er onderzocht moeten worden wat de huidige kennis is van de niet-programmeurs. Dit kan door middel van interviews en een aantal vragen.

Vervolgens zal er onderzocht moeten worden wat een haalbaar moeilijkheidsgraad is en of er baat is van een extra training voor de niet-programmeurs.

Als er sprake is van toegevoegde waarde van een extra training dan zal deze parallel aan het project gegeven worden en iteratief ontwikkeld.

\subsubsection{Welke componenten zullen gemaakt moeten worden?}
Dit zal beginnen met een onderzoek naar de huidige taken vaan programmeur tijdens het maken van een virtuele reality omgeving. Aan de hand hiervan zal er gekeken worden welke taken geautomatiseerd kunnen worden of versimpeld naar virtual scriptring.

Als er een duidelijk beeld is van de benodigde componenten zal er een onderzoek gedaan worden naar bestaande tools die ingezet kunnen worden om de workflow te verbeteren.

Als het duidelijk is welke componenten zelf geschreven moeten worden en hoeverre de niet-programmeurs met VS om kunnen gaan kan er bepaald worden welke abstractie de componenten zo breed inzet baar mogelijk houd terwijl het intuïtief blijft voor de niet-programmeurs

\subsubsection{Is het mogelijk de componenten cross-platform te maken?}
Er zal hier onderzocht moeten worden welke platformen relevant zijn en wat de mogelijkheden voor elk platform is.

Er zal daarna een beslissing gemaakt moeten worden voor elk component of het een alternatieve versie moet hebben of dat cross-platform in het component zelf meegenomen kan worden.
