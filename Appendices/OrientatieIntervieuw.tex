%!TEX root = ../Thesis.tex

\chapter{Oriëntatie Interview}
Interviewer: Mark Arts. Deelnemers: Huib, Danny

\section{Doel}
Het doel van dit interview is om de huidige kennis over gameplay programmering, visuele programmeer talen en logica constructies vast te leggen bij de werknemers van DPI.

Daarnaast word er ook een eerst introductie gemaakt naar blueprints, de visuele programmeer taal die in de Unreal Engine 4 gebruikt word.

Er word geprobeerd antwoord te krijgen op de volgende vragen

\begin{itemize}
	\item Welke ervaring met programmeren hebben de deelnemers
	\item Welke ervaring met UE4 hebben de deelnemers
	\item Welke ervaring met visuele programmeer hebben de deelnemers
	\item Tot hoe verre begrijpen de deelnemers programmeer terminologie (enquête)
\end{itemize}

\section{Interview}
\subsection*{Wat voor 3D projecten hebben jullie tot nu toe gemaakt en wat was jullie taak hierin}
\subsubsection*{Danny}
Voornamelijk het modellen / maken van 3D visualisaties in Max / Maya voor architectuur. Daarnaast een aantal 3D omgevingen in de Unreal Engine 4.
\subsubsection*{Huib}
Voornamelijk het modellen / maken van 3D visualisaties in Max / Maya voor architectuur. Binnen DPI nog niet aan een UE4 project gewerkt. 
\subsection*{Wat is jullie ervaring met Unreal Engine 4}
\subsubsection*{Danny}
Een aantal hobby projecten en binnen DPI het opzetten van een aantal kleine omgevingen in UE4. Lastig om modellen die voor maya / max gebruikt werden te importeren en rekening te houden met performance. De interface van de Engine is wel duidelijk en door ervaring met maya / max was het makkelijk om te beginnen met het bouwen van een omgeving.
\subsubsection*{Huib}
Heeft alleen nog aan hobby projecten gewerkt binnen Unreal maar kon door ervaring met Maya en Max makkelijk aan de slag. Hij had wel moeite met het gebruikt van Blueprints in demo’s en vond het vaak overweldigend.
\subsection*{Wat weten jullie over programmeren}
\subsubsection*{Danny}
Geen ervaring met programmeren en weet er weinig over.
Huib
Heeft CMD gestudeerd en heeft tijdens zijn studie ervaring opgedaan met website’s programmeren. Hij snapt hoe code werkt en wat je er mee kan doen maar zou niet c++ voor een game kunnen programmeren.
\subsection*{Wat voor ervaring hebben jullie met visuele programeer talen}
Danny
Geen ervaring.
Huib
Heeft Blueprints geprobeerd maar daarnaast geen ervaring.

\section[Enquête]{Tot hoe verre begrijpen de deelnemers programmeer terminologie (enquête)}
Er is door zowel Huib en Danny een vragenlijst ingevuld met de volgende introductie:

Omschrijf in eigen worden wat jij denkt dat de volgende begrippen bekennen. Als het begrip onbekend is omschrijf wat jij denkt dat het zou zijn.

\subsection*{Danny}
\subsubsection*{Branch (conditional statement)}
Gok: ik denk een vertakking in verschillende nodes?
\subsubsection*{For loop}
Gok: ik denk een loop in een script, bijvoorbeeld een walkcycle
\subsubsection*{Switch statement}
Gok: True of false switch?
\subsubsection*{Integer}
geen idee
\subsubsection*{Float}
Gok: gravity? geen idee.
\subsubsection*{Boolean}
objecten die met elkaar worden gecombineerd en kan worden bepaald of er gesubstract wordt.
\subsubsection*{Vectors}
De punten van een object (uiteindes)
\subsubsection*{Transform}
Geometry aanpassen in onderandere scalen
\subsubsection*{Struct (structur)}
Gok: hoe een object is opgebouwd? quad en tris?
\subsubsection*{Array}
Op deze manier wordt er iets geduplicate
\subsubsection*{Class}
verschillende blueprints, zoals firstpersonmode of playercontroller.
\subsubsection*{Object Type}
wat voor object, static mesh of een BSP
\subsubsection*{Propertie}
instellingen van bepaalde actors.
\subsubsection*{Reference (pointer)}
Je kan volgens mij objecten converteren naar triggerboxen.
\subsubsection*{Cast}
Gok: ik heb het weleens langs zien komen, maar volgens mij zijn het nodes in het script van blueprints
\subsubsection*{Actor}
eigenlijk alles wat in de scene wordt gezet.
\subsubsection*{Component}
Gok: heeft iets met de actors te maken, ik denk bepaalde instellingen daarin.

\subsection*{Huib}
\subsubsection*{Branch (conditional statement)}
Keus uit meerdere mogelijkheden / inputs
\subsubsection*{For loop}
Iets doen wanneer iets een bepaalde waarde heeft...
\subsubsection*{Switch statement}
Schakelaar met meer dan 2 mogelijkheden.
\subsubsection*{Integer}
Heel getal
\subsubsection*{Float}
Getal met decimalen
\subsubsection*{Boolean}
True / False
\subsubsection*{Vectors}
Punt in 3D space (x,y,z coördinaat)
\subsubsection*{Transform}
Wat dit in de context van Unreal is weet ik niet, ik ken het als iets dat te maken heeft met de positie, rotatie en schaal van objecten.
\subsubsection*{Struct (structur)}
Constructie van een element.
\subsubsection*{Array}
Een reeks
\subsubsection*{Class}
Een groep objecten met dezelfde eigenschappen.
\subsubsection*{Object Type}
Een soort object, mesh, light, etc.
\subsubsection*{Propertie}
Eigenschap van een object.
\subsubsection*{Reference (pointer)}
Gok: Verwijzing naar iets
\subsubsection*{Cast}
Gok: Iets versturen
\subsubsection*{Actor}
Een object dat ergens op kan reageren.
\subsubsection*{Component}
Een bouwblok