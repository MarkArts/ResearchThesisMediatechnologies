%!TEX root = ../Thesis.tex

\chapter{Oriëntatie Interview}
\label{appendix:oreintatieintervieuw}
\lhead{}
Interviewer: Mark Arts. Deelnemers: Huib, Danny

\section{Doel}
Het doel van dit interview is om de huidige kennis over gameplay programmering, visuele programmeer talen en logica constructies vast te leggen bij de werknemers van DPI.

Daarnaast wordt er ook een eerst introductie gemaakt naar Blueprints, de visuele programmeer taal die in de Unreal Engine 4 gebruikt wordt.

Er wordt geprobeerd antwoord te krijgen op de volgende vragen:

\begin{itemize}
	\item Welke ervaring met programmeren hebben de deelnemers
	\item Welke ervaring met UE4 hebben de deelnemers
	\item Welke ervaring met visueel programmeren hebben de deelnemers
	\item Tot hoe verre begrijpen de deelnemers programmeer terminologie (enquête)
\end{itemize}

\section{Interview}
\label{appendix:oreintatieintervieuw:interview}
\subsection*{Wat voor 3D projecten hebben jullie tot nu toe gemaakt en wat was jullie taak hierin}
\subsubsection*{Danny}
Voornamelijk het modellen / maken van 3D visualisaties in Max / Maya voor architectuur. Daarnaast een aantal 3D omgevingen in de Unreal Engine 4.
\subsubsection*{Huib}
Voornamelijk het modellen / maken van 3D visualisaties in Max / Maya voor architectuur. Binnen DPI nog niet aan een UE4 project gewerkt. 
\subsection*{Wat is jullie ervaring met Unreal Engine 4}
\subsubsection*{Danny}
Een aantal hobby projecten en binnen DPI het opzetten van een aantal kleine omgevingen in UE4. Lastig om modellen die voor maya / max gebruikt werden te importeren en rekening te houden met performance. De interface van de Engine is wel duidelijk en door ervaring met maya / max was het makkelijk om te beginnen met het bouwen van een omgeving.
\subsubsection*{Huib}
Heeft alleen nog aan hobby projecten gewerkt binnen Unreal maar kon door ervaring met Maya en Max makkelijk aan de slag. Hij had wel moeite met het gebruik van Blueprints in demo’s en vond het vaak overweldigend.
\subsection*{Wat weten jullie over programmeren}
\subsubsection*{Danny}
Geen ervaring met programmeren en weet er weinig over.
\subsubsection*{Huib}
Heeft CMD gestudeerd en heeft tijdens zijn studie ervaring opgedaan met website’s programmeren. Hij snapt hoe code werkt en wat je er mee kan doen maar zou niet C++ voor een game kunnen programmeren.
\subsection*{Wat voor ervaring hebben jullie met visuele programeer talen}
\subsubsection*{Danny}
Geen ervaring.
\subsubsection*{Huib}
Heeft Blueprints geprobeerd maar daarnaast geen ervaring.

\section[Enquête]{Tot hoeverre begrijpen de deelnemers programmeer terminologie (enquête)}
\label{appendix:oreintatieintervieuw:enquete}
Er is door zowel Huib en Danny een vragenlijst ingevuld met de volgende introductie:

Omschrijf in eigen woorden wat jij denkt dat de volgende begrippen betekenen. Als het begrip onbekend is omschrijf wat jij denkt dat het betekent.

\subsection*{Danny}
\textbf{Branch (conditional statement)} \newline
Gok: ik denk een vertakking in verschillende nodes? \newline
\textbf{For loop} \newline
Gok: ik denk een loop in een script, bijvoorbeeld een walkcycle \newline
\textbf{Switch statement} \newline
Gok: True of false switch? \newline
\textbf{Integer} \newline
geen idee \newline
\textbf{Float} \newline
Gok: gravity? geen idee. \newline
\textbf{Boolean} \newline
objecten die met elkaar worden gecombineerd en kan worden bepaald of er gesubstract wordt. \newline
\textbf{Vectors} \newline
De punten van een object (uiteindes) \newline
\textbf{Transform} \newline
Geometry aanpassen in onderandere scalen \newline
\textbf{Struct (structur)} \newline
Gok: hoe een object is opgebouwd? quad en tris? \newline
\textbf{Array} \newline
Op deze manier wordt er iets geduplicate \newline
\textbf{Class} \newline
verschillende blueprints, zoals firstpersonmode of playercontroller. \newline
\textbf{Object Type} \newline
wat voor object, static mesh of een BSP \newline
\textbf{Propertie} \newline
instellingen van bepaalde actors. \newline
\textbf{Reference (pointer)} \newline
Je kan volgens mij objecten converteren naar triggerboxen. \newline
\textbf{Cast} \newline
Gok: ik heb het weleens langs zien komen, maar volgens mij zijn het nodes in het script van Blueprints \newline
\textbf{Actor} \newline
eigenlijk alles wat in de scene wordt gezet. \newline
\textbf{Component} \newline
Gok: heeft iets met de actors te maken, ik denk bepaalde instellingen daarin.

\subsection*{Huib}
\textbf{Branch (conditional statement)} \newline
Keus uit meerdere mogelijkheden / inputs \newline
\textbf{For loop} \newline
Iets doen wanneer iets een bepaalde waarde heeft... \newline
\textbf{Switch statement} \newline
Schakelaar met meer dan 2 mogelijkheden. \newline
\textbf{Integer} \newline
Heel getal \newline
\textbf{Float} \newline
Getal met decimalen \newline
\textbf{Boolean} \newline
True / False \newline
\textbf{Vectors} \newline
Punt in 3D space (x,y,z coördinaat) \newline
\textbf{Transform} \newline
Wat dit in de context van Unreal is weet ik niet, ik ken het als iets dat te maken heeft met de positie, rotatie en schaal van objecten. \newline
\textbf{Struct (structur)} \newline
Constructie van een element. \newline
\textbf{Array} \newline
Een reeks \newline
\textbf{Class} \newline
Een groep objecten met dezelfde eigenschappen. \newline
\textbf{Object Type} \newline
Een soort object, mesh, light, etc. \newline
\textbf{Propertie} \newline
Eigenschap van een object. \newline
\textbf{Reference (pointer)} \newline
Gok: Verwijzing naar iets \newline
\textbf{Cast} \newline
Gok: Iets versturen \newline
\textbf{Actor} \newline
Een object dat ergens op kan reageren. \newline
\textbf{Component} \newline
Een bouwblok